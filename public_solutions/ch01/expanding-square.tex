The area is $a=\ell^2=(1+\alpha T)^2\ell_\zu{o}^2$. To make this into something we know how
to differentiate, we need to square out the expression involving $T$, and make it into something
that is expressed explicitly as a polynomial:
\begin{equation*}
  a=\ell_\zu{o}^2+2\ell_\zu{o}^2\alpha T+\ell_\zu{o}^2\alpha^2T^2
\end{equation*}
Now this is just like problem \ref{hw:diff-symbolic-const}, except that the constants superficially
look more complicated. The result is
\begin{align*}
  \frac{\der a}{\der T} &=2\ell_\zu{o}^2\alpha +2\ell_\zu{o}^2\alpha^2T \\
          &=2\ell_\zu{o}^2\left(\alpha +\alpha^2T\right) \qquad .
\end{align*}

We expect the units of the result to be area per unit temperature, e.g., degrees per square meter.
This is a little tricky, because we have to figure out what units are implied for the constant
$\alpha$. Since the question talks about $1+\alpha T$, apparently the quantity $\alpha T$ is unitless.
(The 1 is unitless, and you can't add things that have different units.) Therefore the units
of $\alpha$ must be ``per degree,'' or inverse degrees. It wouldn't make sense to add $\alpha$ and
$\alpha^2T$ unless they had the same units (and you can check for yourself that they do), so
the whole thing inside the parentheses must have units of inverse degrees. Multiplying by the
$\ell_\zu{o}^2$ in front, we have units of area per degree, which is what we expected.
