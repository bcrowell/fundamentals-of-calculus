All five of these can be done using l'H\^{o}pital's rule:

\begin{align*}
  \lim_{s\rightarrow 1} & \frac{s^3-1}{s-1} = \lim \frac{3s^2}{1} = 3 \\
  \lim_{\theta\rightarrow 0} & \frac{1-\cos\theta}{\theta^2} = \lim \frac{\sin\theta}{2\theta} = \lim \frac{\cos\theta}{2} = \frac{1}{2} \\
  \lim_{x\rightarrow \infty} & \frac{5x^2-2x}{x} = \lim \frac{10x-2}{1} = \infty \\
  \lim_{n\rightarrow \infty} & \frac{n(n+1)}{(n+2)(n+3)} = \lim \frac{n^2+\ldots}{n^2+\ldots}  = \lim \frac{2n+\ldots}{2n+\ldots} = \lim \frac{2}{2} = 1 \\
  \lim_{x\rightarrow \infty} & \frac{ax^2+bx+c}{dx^2+ex+f} = \lim \frac{2ax+\ldots}{2dx+\ldots} = \lim \frac{2a}{2d} = \frac{a}{d}
\end{align*}
In examples 2, 4, and 5, we differentiate more than once in order to get an expression that can be evaluated by
substitution. In 4 and 5, \ldots represents terms that we anticipate will go away after the second differentiation.
Most people probably would not bother with l'H\^{o}pital's rule for 3, 4, or 5, being content merely to observe
the behavior of the highest-order term, which makes the limiting behavior obvious. Examples 3, 4, and 5 can also
be done rigorously without l'H\^{o}pit rule, by algebraic manipulation;
we divide on the top and bottom by the highest power of the variable, giving an expression that is no longer
an indeterminate form $\infty/\infty$.
