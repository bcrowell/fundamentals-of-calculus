Since $f$, $g$, and $s$ are smooth and defined everywhere, any extrema they possess occur at places where their
derivatives are zero. The converse is not necessarily true, however; a place where the derivative is zero could
be a point of inflection. The derivative is additive, so if \emph{both} $f$ and $g$ have zero derivatives at a certain
point, $s$ does as well. Therefore in most cases, if $f$ and $g$ both have an extremum at a point, so will $s$.
However, it could happen that this is only a point of inflection for $s$, so in general, we can't conclude anything
about the extrema of $s$ simply from knowing where the extrema of $f$ and $g$ occur.

Going the other direction, we certainly can't infer anything about extrema of $f$ and $g$ from knowledge of $s$ alone.
For example, if $s(x)=x^2$, with a minimum at $x=0$, that tells us very little about $f$ and $g$. We could have, for example,
$f(x)=(x-1)^2/2-2$ and $g(x)=(x+1)^2/2+1$, neither of which has an extremum at $x=0$.
