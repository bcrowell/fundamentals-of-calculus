If $R_1$ is finite and $R_2$ approaches infinity, then $1/R_2$ is approaches zero.
$1/R_1+1/R_2$ approaches $1/R_1$, and the combined resistance $R$ approaches
from $R_1$. Physically, the second pipe is blocked or too thin to carry any significant flow,
so it's as though it weren't present.

If $R_1$ is finite and $R_2$ gets very small, then $1/R_2$ gets very big, $1/R_1+1/R_2$ is
dominated by the second term, and the result is basically the same as $R_2$.
It's so easy for water to
flow through $R_2$ that $R_1$ might as well not be present. In the context of electrical circuits
rather than water pipes, this is known as a short circuit.
