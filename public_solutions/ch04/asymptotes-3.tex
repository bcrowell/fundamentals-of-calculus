The function
\begin{equation*}
  f(x) = \left(\frac{x^2+1}{x^2+2}-\frac{x^2+3}{x^2+4}\right)^{-1} 
\end{equation*}
is not given in the form of a rational function, and the most 
straightforward thing to do here would be simply to change it into
that form. Before we do that, however, we could look for values of
$x$ at which the quantity inside the parentheses would go to zero;
these would be the vertical asymptotes. Setting the denominator
equal to zero gives $(x^2+1)(x^2+4)=(x^2+2)(x^2+3)$, which simplifies
to $4=6$. There are no solutions, and therefore the functoin has no
vertical asymptotes.

Going ahead and recasting it as a rational function, we first need to
put the two terms over a common denominator. This gives
\begin{equation*}
  f(x) = \left(\frac{(x^2+1)(x^2+4)-(x^2+2)(x^2+3)}{(x^2+2)(x^2+4)}\right)^{-1}  
qquad ,
\end{equation*}
which simplifies to
\begin{align*}
  f(x) &= \left(\frac{-2}{(x^2+2)(x^2+4)}\right)^{-1}   \\
       &= -\frac{1}{2}(x^2+2)(x^2+4) 
qquad .
\end{align*}
We now see that the exotic-looking function was in fact just a polynomial in
disguise. Polynomials don't have horizontal or vertical asymptotes.
