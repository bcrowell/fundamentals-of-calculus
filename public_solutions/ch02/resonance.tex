We can make life a lot easier by observing that the function $s(f)$ will be maximized when the
expression inside the square root is minimized. Also, since $f$ is squared every time it occurs,
we can change to a variable $x=f^2$,
and then once the optimal value of $x$ is found we can take its square root in order to find
the optimal $f$. The function to be optimized is then
\begin{equation*}
  a(x-f_\zu{o}^2)^2+bx \qquad .
\end{equation*}
Differentiating this and setting the derivative equal to zero, we find
\begin{equation*}
  2a(x-f_\zu{o}^2)+b = 0 \qquad,
\end{equation*}
which results in $x=f_\zu{o}^2-b/2a$, or
\begin{equation*}
  f = \sqrt{f_\zu{o}^2-b/2a} \qquad,
\end{equation*}
(choosing the positive root, since $f$ represents a frequencies, and frequencies are positive by definition).
Note that the quantity inside the square root involves the square of a frequency, but then we take its
square root, so the units of the result turn out to be frequency, which makes sense. We can see that
if $b$ is small, the second term is small, and the maximum occurs very nearly at
$f_\zu{o}$. 

There is one subtle issue that was glossed over above, which is that the graph on page \pageref{fig:resonance}
shows \emph{two} extrema: a minimum at $f=0$ and a maximum at $f>0$. What happened to the $f=0$ minimum?
The issue is that I was a little sloppy with the change of variables. Let $I$ stand for the quantity inside
the square root in the original expression for $s$. Then by the chain rule,
\begin{equation*}
  \frac{\der s}{\der f} =   \frac{\der s}{\der I} \cdot \frac{\der I}{\der x} \cdot \frac{\der x}{\der f} .
\end{equation*}
We looked for the place where $\der I/\der x$ was zero, but $\der s/\der f$ could also be zero if one of the
other factors was zero. This is what happens at $f=0$, where $\der x/\der f=0$.
