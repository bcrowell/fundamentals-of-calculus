A piston in a car's engine is connected to the crankshaft through a
piston rod. As the crankshaft spins at a constant rate, the velocity
of the piston in and out of the cylinder may be approximated by
a function
\begin{equation*}
  v = A\cos\omega t+B\cos 2\omega t \qquad ,
\end{equation*}
where $\omega$ (Greek letter ``omega,'' which makes the ``o'' sound)
is the number of radians per second at which the crankshaft is rotating,
and $A$ and $B$ are constants that depend on the length of the piston
rod and the radius of the circle traveled by the piston pin. Note that expressions
of the form $\sin xy$ are normally to be read as $\sin(xy)$; if the intended meaning
had been $(\sin x)y$, then one would normally have written it as $y\sin x$.\\
(a) Infer the units of $A$ and $B$. (The units of $\omega$ are simply inverse
seconds, $\sunit^{-1}$.)\hwendpart
(b) Find the piston's position $x$ as a function of time.\answercheck\hwendpart
(c) Give a physical interpretation of the constant of integration occurring in
your answer to part b.\hwendpart
(d) Check that your answer to part b has units that make sense.\hwendpart
(e) Check your answer by differentiating it.
