\emph{See the instructions on p.~\pageref{hw:longbow}.} In a gasoline-burning car engine, the exploding air-gas
mixture makes a force on the piston, and the force tapers off as the piston expands, allowing the
gas to expand.
A not-so-bad approximation is that the force is given by $F=k/x$,
where $x$ is the position of the piston.
(a) Infer the units of $k$.
(b) Find the
work done on the piston as it travels from $x=a$ to $x=b$.
(c) Show that the result of part b can be reexpressed so that it depends only
on the ratio $b/a$. This ratio is known as the compression ratio of the engine.
(d) Check that the units of your result in part c make sense.
\answercheck
