An electric meter installed outside your household measures the
flow of electric current $I$. If you turn on a lamp, $I$ increases,
and if you turn it back off again, $I$ goes back down.
The cost $C$ of the electricity is also a function of time; it grows
until it's time for the electric company to bill you.
Consider the following
two proposed relations between these variables.
\begin{align*}
  I &= k\frac{\der C}{\der t} \\
  I &= k\int_{t_1}^{t_2} C \: \der t
\end{align*}
Here $k$ is a constant. 
Use one of the methods of section \ref{subsec:to-differentiate-or-to-integrate},
p.~\pageref{subsec:to-differentiate-or-to-integrate}, to determine which of these
makes sense.
