In example \ref{eg:lottery}, p.~\pageref{eg:lottery}, we found the
maximum amount that a person should be willing to pay for a lottery
ticket, given a certain utility function. We assumed the utility function
to be concave down, which is usually realistic, for the reasons
discussed in the example. But there can also be cases where the utility
function is concave up. Suppose that Sally has cancer and no health insurance.
She can only survive if she gets expensive treatment, which she can't presently
afford. A small amount of money does her very little good, except that it slightly
reduces the amount she still needs to get together for the treatment.
In this situation, it might make sense to posit a concave-up utility function,
such as $u(x)=x^2$, in the notation of the previous example.
Redo the example with this utility function.\answercheck
