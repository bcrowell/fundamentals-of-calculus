A slice of pie subtending an angle $\theta$ (in radians) is cut from
a pie of radius $r$. (You may wish to review the definition of radian
measure, section \ref{subsec:radian-measure}, p.~\pageref{subsec:radian-measure}.)\\
(a) Find the perimeter $P$ of the slice, i.e., the sum of
the lengths of its two straight sides plus the arc
length of the curved side.\answercheck\hwendpart
(b) Find the area $A$ of the slice.\answercheck\hwendpart
(c) Suppose we want to make a pie-slice shape with the minimum possible perimeter for a fixed area.
(The radius $r$ is \emph{not} fixed.)
Use your answer to part b to eliminate $r$ from part a, and
find the perimeter as a function of $A$ and $\theta$.\answercheck\hwendpart
(d) Find the value of $\theta$ that minimizes the perimeter, treating $A$ as a constant.\answercheck\hwendpart
