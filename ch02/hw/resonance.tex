When you tune in a radio station using an old-fashioned rotating dial you don't have to be
exactly tuned in to the right frequency in order to get the station. If you did, the
tuning would be infinitely sensitive, and you'd never be able to receive any signal at
all! Instead, the tuning has a certain amount of ``slop'' intentionally designed into it.
The strength of the received signal $s$ can be expressed in terms of the dial's setting
$f$ by a function of the form
\begin{equation*}
  s = \frac{1}{\sqrt{a(f^2-f_\zu{o}^2)^2+bf^2}} \qquad ,
\end{equation*}
where $a$, $b$, and $f_\zu{o}$ are constants. The constant $b$ relates to the amount of slop.
This functional form is in fact very general, and
is encountered in many other physical contexts. The graph  shows an example of the kind of
bell-shaped that results curve.
Find the frequency $f$ at which the maximum response occurs, and show that if $b$ is small,
the maximum occurs close to, but not exactly at, $f_\zu{o}$.
