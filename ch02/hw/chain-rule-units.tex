In section \ref{subsec:composition} on p.~\pageref{subsec:composition},
we expressed the chain rule without the Leibniz notation, writing a function $f$ defined by
$f(x)=g(h(x))$. Suppose that you're trying to remember the rule, and two of the possibilities
that come to mind are $f'(x)=g'(h(x))$ and $f'(x)=g'(h(x))h(x)$. Show that neither of these
can possibly be right, by considering the case where $x$ has units. You may find it helpful
to convert both expressions back into the Leibniz notation.
