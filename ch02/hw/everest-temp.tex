The following table shows the barometric pressure $P$ and average July temperature $T$
for the summit of Mount Everest and the city of Wenzhou, China, which is at the
same latitude.\\

\begin{tabular}{lll}
           & pressure (kPa) & temperature (\degcunit) \\
Wenzhou    & 101            & $+29$ \\
Everest    & 38             & $-16$
\end{tabular}

A physical model predicts the following relationship between
these two variables:
\begin{equation*}
  T = T_\zu{o} + cP^{2/7}
\end{equation*}
Here $c$ is a constant and $T_\zu{o}=-273\degcunit$ is a constant that converts
from degrees Celsius to a temperature scale based on absolute zero.\\
(a) Estimate $c$ from the data at Wenzhou.\answercheck\hwendpart
(b) $T$ is a complicated nonlinear function of $P$, and for some purposes, such
as mental estimation,
a linear approximation might be more convenient to work with.
Find the equation of the tangent line to this function at the point
representing the conditions at Wenzhou, and use this equation to
calculate the expected temperature at the summit of Everest.
This is quite a long extrapolation.
How good an approximation is it?\answercheck\hwendpart
