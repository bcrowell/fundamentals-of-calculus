Let $f(x)=x^n$, where $n$ is an integer greater than or equal to 1,
and suppose that we want to evaluate $f'(1)$ directly using the definition of the limit,
i.e., using the brute-force technique of example \ref{eg:x-squared-with-limits},
p.~\pageref{eg:x-squared-with-limits}. This will involve multiplying out the
expression $(1+\Delta x)^n-1$, after which we end up throwing away everything except for the
lowest-order nonvanishing term (i.e., the term with $\Delta x$ to the first power).
All we really need is the coefficient of this term, which in example
\ref{eg:x-squared-with-limits} was 2.
For a particular value of $n$, we could just go ahead and multiply out this expression, but suppose we would rather
prove the result for all $n$. This requires that we prove a general result for
the coefficient of the linear term in the expression $(1+\Delta x)^n$. Such a coefficient is called
a binomial coefficient. 
Proof by induction was introduced in section \ref{subsec:natural-powers}, p.~\pageref{induction}.
Use a proof by induction to show that the binomial coefficient we're talking about equals $n$.
